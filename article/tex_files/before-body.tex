% Capa
\thispagestyle{empty}
\pagenumbering{roman}

\begin{center}
$institution$
\end{center}


\vspace{4.0cm}

\begin{center}
    \textbf{$title$}
\end{center}

\vspace{4.0cm}

\begin{center}
$author$

Matrícula: 99079
\end{center}


\vspace{4cm}

\begin{center}

ORIENTADOR(A): $advisor$.
\end{center}

\vspace{4cm}

\begin{center}

$city$ - $state$

$month$ - $year$
    
\end{center}

\newpage
% Folha de aprovação
\thispagestyle{empty}

\begin{center}
$author-upper$
\end{center}

\vspace{1.0cm}

\begin{center}
    \textbf{$title$}
\end{center}

\vspace{3.0cm}

\begin{adjustwidth}{7.5cm}{0cm}
    \justifying
    $confirm_text$
\end{adjustwidth}

\vspace{1.0cm}

APROVADA: $confirm_date$

\vspace{1.0cm}

\begin{minipage}{0.45\textwidth}
    \centering
    \rule{6cm}{0.4pt} \\
    $examiner1$
\end{minipage}%
\hspace{0.1\textwidth}
\begin{minipage}{0.45\textwidth}
    \centering
    \rule{6cm}{0.4pt} \\
    $examiner2$
\end{minipage}

\vspace{2.0cm}

\centering
\rule{6cm}{0.4pt} \\
$advisor-name$ \\
(Orientador)

\vspace{2.0cm}

\begin{center}

$city$ - $state$

$month$ - $year$
    
\end{center}

\newpage
% Folha de responsabilidade
\thispagestyle{empty}

\begin{invisible}
.
\end{invisible}

\vspace{28\baselineskip}

\textit{As opiniões expressas neste trabalho são de exclusiva responsabilidade do autor.}


\newpage
\pagestyle{fancy}
% Dedicatoria

\begin{invisible}
.
\end{invisible}

\vspace{28\baselineskip}

\begin{adjustwidth}{7.5cm}{0cm}
    \justifying
    \singlespacing
    $dedication$
\end{adjustwidth}

\newpage
\pagestyle{fancy}

\newcommand{\resumo}{%
\titleformat{\section}[block]{\bfseries\filcenter}{}{1em}{}
}
\resumo

\section*{AGRADECIMENTO}
\justify

\begin{invisible}
.
\end{invisible}

Gostaria de expressar minha profunda gratidão às pessoas e instituições que contribuíram significativamente para a conclusão desta monografia.

Agradeço ao meu orientador, Professor Igor Santos Tupy, por sua orientação perspicaz, paciência e incentivo ao longo deste processo. Sua experiência e empenho foram fundamentais para o desenvolvimento deste trabalho.

Agradeço aos membros da banca avaliadora por dedicarem seu tempo para revisar e avaliar este trabalho.

À minha família, em especial a minha mãe Marlene, meu pai Emerson, meu irmão Paulo e meu padrasto Cidemir, pelo apoio incondicional, encorajamento e compreensão durante este período desafiador.

À minha noiva, Natália Cristina, que esteve sempre ao meu lado, contribuiu em todas etapas da minha formação e não mediu esforços para me manter feliz e resiliente.

Ao meu tio, Vinícius, que foi um grande parceiro nesse período, dividiu quarto e cuidou de mim desde o primeiro dia na UFV. Meu muito obrigado.

Aos meus amigos, colegas de classe e de trabalho, obrigado por compartilharem ideias, sorrisos e por tornarem esta jornada acadêmica mais enriquecedora.

Agradeço à Universidade Federal de Viçosa por fornecer o ambiente propício para realização deste trabalho.

Agradeço ao IMPA e ao Instituto TIM pelo apoio financeiro em minha trajetória acadêmica, esse apoio foi fundamental para que eu pudesse concluir o bacharelado em Ciências Econômicas.

Por último, agradeço a todos que, de alguma forma, contribuíram para este trabalho, direta ou indiretamente. Seus esforços não passaram despercebidos e são imensamente apreciados.

\newpage
% Resumo
\pagestyle{fancy}

\section*{RESUMO}


\begin{spacing}{1.0}
\justify

\setlength{\parindent}{0pt}

PANGRACIO SILVA, Flávio Hugo; Universidade Federal de Viçosa, Dezembro/2023. \textbf{As centralidades financeiras no espaço urbano: uma análise espacial empírica e exploratória do setor bancário no município de São Paulo, no ano de 2010.} Orientador: Prof. Dr. Igor Santos Tupy.

O presente estudo investiga a centralidade financeira urbana no municipio de São Paulo em 2010, empregando métodos de análise regional e referenciais pós-keynesianos para avaliar o setor bancário. Por meio de dados da estatística bancária (ESTBAN), do CENSO 2010 e da RAIS, construiu-se uma base de dados georreferenciada que permitiu análises espaciais exploratórias e empíricas robustas. As análises revelaram uma concentração significativa de agências e crédito no centro da cidade, além de uma profunda desiguldade na qualidade de crédito ofertada entre os distritos periféricos e centrais. Por meio dos resultados do modelo econométrico espacial, pôde-se inferir que a qualidade financeira local e dos 5 vizinhos mais próximos influencia positivamente a renda \textit{per capita} dos distritos.

\textbf{Palavras-chave:} $keywords$

\setlength{\parindent}{15pt}

\end{spacing}

\newpage
% Lista de figuras
\renewcommand{\listfigurename}{LISTA DE FIGURAS}
\pagestyle{fancy}
\listoffigures

\newpage
% Lista de tabelas
\renewcommand{\listtablename}{LISTA DE TABELAS}
\pagestyle{fancy}
\listoftables

\newpage
% Sumário
\renewcommand{\contentsname}{SUMÁRIO}
\pagestyle{fancy}
\tableofcontents

\titleformat{\section}{\normalsize\bfseries}{\thesection.}{1em}{}
\newpage
\pagenumbering{arabic}
